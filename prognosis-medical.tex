\documentclass{beamer}

\mode<presentation> {
	
	% The Beamer class comes with a number of default slide themes
	% which change the colors and layouts of slides. Below this is a list
	% of all the themes, uncomment each in turn to see what they look like.
	
	%\usetheme{default}
	%\usetheme{AnnArbor}
	%\usetheme{Antibes}
	%\usetheme{Bergen}
	%\usetheme{Berkeley}
	%\usetheme{Berlin}
	%\usetheme{Boadilla}
	%\usetheme{CambridgeUS}
	%\usetheme{Copenhagen}
	%\usetheme{Darmstadt}
	%\usetheme{Dresden}
	%\usetheme{Frankfurt}
	%\usetheme{Goettingen}
	%\usetheme{Hannover}
	%\usetheme{Ilmenau}
	%\usetheme{JuanLesPins}
	%\usetheme{Luebeck}
	\usetheme{Madrid}
	%\usetheme{Malmoe}
	%\usetheme{Marburg}
	%\usetheme{Montpellier}
	%\usetheme{PaloAlto}
	%\usetheme{Pittsburgh}
	%\usetheme{Rochester}
	%\usetheme{Singapore}
	%\usetheme{Szeged}
	%\usetheme{Warsaw}
	
	% As well as themes, the Beamer class has a number of color themes
	% for any slide theme. Uncomment each of these in turn to see how it
	% changes the colors of your current slide theme.
	
	%\usecolortheme{albatross}
	%\usecolortheme{beaver}
	%\usecolortheme{beetle}
	%\usecolortheme{crane}
	%\usecolortheme{dolphin}
	%\usecolortheme{dove}
	%\usecolortheme{fly}
	%\usecolortheme{lily}
	%\usecolortheme{orchid}
	%\usecolortheme{rose}
	%\usecolortheme{seagull}
	%\usecolortheme{seahorse}
	%\usecolortheme{whale}
	%\usecolortheme{wolverine}
	
	%\setbeamertemplate{footline} % To remove the footer line in all slides uncomment this line
	%\setbeamertemplate{footline}[page number] % To replace the footer line in all slides with a simple slide count uncomment this line
	
	%\setbeamertemplate{navigation symbols}{} % To remove the navigation symbols from the bottom of all slides uncomment this line
}

\usepackage{graphicx} % Allows including images
\usepackage{booktabs} % Allows the use of \toprule, \midrule and \bottomrule in tables 

\usepackage[T1]{fontenc}
\usepackage[utf8]{inputenc}
\setbeamertemplate{caption}[numbered]
\newcommand{\C}{\mathbb{C}}
\newcommand{\R}{\mathbb{R}}
\newcommand{\Q}{\mathbb{Q}}
\newcommand{\Z}{\mathbb{Z}}
\newcommand{\N}{\mathbb{N}}
\newcommand{\p}{\mathbb{P}}
\newcommand{\E}{\mathbb{E}}
\usepackage{graphicx}
\usepackage{amssymb}
\usepackage{setspace}
\usepackage[toc,page]{appendix}
\usepackage{epstopdf}
\usepackage{latexsym}
\usepackage{amstext}
\usepackage{lmodern}
\usepackage{amsmath}
\usepackage{bbm}
\usepackage{amsfonts}
\usepackage{url}
\usepackage{bm}
\usepackage{mathrsfs}
\usepackage{mathtools}
\usepackage{float}
%\usepackage{hyperref} give reference hyperlink 
%\usepackage{setspace}
\usepackage{indentfirst}
\usepackage{multirow}
\usepackage{color}
\usepackage{mathtools}
% packages from template
\usepackage{amsmath,amsthm,amssymb,amsfonts}
\usepackage[width=.9\textwidth]{caption}
\usepackage{mathrsfs}
\usepackage{graphicx}
\newcommand{\indep}{\rotatebox[origin=c]{90}{$\models$}}
\usepackage{textgreek}
\usepackage{bbold}
\usepackage{subcaption}
\usepackage{natbib}
\usepackage{verbatim}
\usepackage{soul}
\usepackage[utf8]{inputenc}
\usepackage[algo2e,ruled,vlined]{algorithm2e} 
\usepackage{fancyvrb}


\newtheorem{proposition}[theorem]{Proposition}

\newcommand{\M}{\mathbf{M}}
\newcommand{\rank}{\mathrm{rank}}
\newcommand{\rep}{\mathrm{rep}}
\newcommand{\PR}{\text{Pr}}
\newcommand{\pkg}[1]{{\fontseries{b}\selectfont #1}}
%\newcommand\norm[1]{\left\lVert#1\right\rVert}
\newcommand{\bs}[1]{\pmb{#1}}
\newcommand{\mb}[1]{\mathbf{#1}}
\DeclareMathOperator*{\argmin}{arg\,min}
\DeclarePairedDelimiter{\ceil}{\lceil}{\rceil}
\DeclarePairedDelimiterX{\norm}[1]{\lVert}{\rVert}{#1}
\allowdisplaybreaks
%=============================================================================
% prelude
%=============================================================================
\def\mathLarge#1{\mbox{\LARGE $#1$}}
\usepackage{soul}


\title[]{Topics in Image-based Prognosis}
\author[Hongda Zhang]{Hongda Zhang}
\institute{Nanjing University}
\date{\today}



\begin{document}
	\begin{frame}
		\titlepage
	\end{frame}
	
	\begin{frame}
		\frametitle{Contents}
		\begin{enumerate}
			\item Whole slide images based cancer survival predicting using attention guided deep multiple instance learning networks
			\item Prediction cancer outcomes from histology and genomics using convolutional networks
		\end{enumerate}
		\nocite{*}
	\end{frame}
	
	\begin{frame}
		\frametitle{DeepAttnMISL}
		Contributions
		
		The Deep Attention Multiple-Instance Survival Learning (DeepAttnMISL) is proposed to make accurate prognosis for cancer patients using whole slide images. The main contributions are shown below:
		\begin{itemize}
			\item The proposed multiple instance deep neural network first extract instance-level features from a number of patches through a Siamese MIL-based network. Then features from multiple instances are aggregated according to the attentions. The multiple instance framework solve the problem that each patient have many image patches used for prognosis. Using each patch as if it is from a separate individual may cause bias since the number of patches of different patients varies. In contrast, the multiple instance framework aggregates features from multiple patches to one patient-level feature. Moreover, the attention based feature aggregation is more flexible than fixed pooling operations, e.g. max pooling, and helps make better prognosis. 
		\end{itemize}
	\end{frame}
	
	\begin{frame}
		\frametitle{DeepAttnMISL}
		Contributions (cont.)
		
		\begin{itemize}
			\item The proposed model is useful for finding prognosis relevant patches from the whole slide images. The relevance of the instances are compared using the calculated attentions. The set a patches in an instance with greater absolute value of attention are considered more relevant to prognosis.
			\item Extensive experiments are conducted on two large datasets to access the performance of the proposed framework. One dataset is from National Lung Screening Trial (NLST) and the other is from the Molecular and Cellular Oncology (MCO) study.
		\end{itemize}
	\end{frame}
	
	\begin{frame}[allowframebreaks]
		\begin{singlespace}
			\interlinepenalty=10000	% prevents bib items from splitting across pages
			\bibliography{prognosis-medical}
			\bibliographystyle{apalike}
		\end{singlespace}
	\end{frame}
	
\end{document} 